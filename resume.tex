%-------------------------------------------------------------------------------
%                             ADDITIONAL PACKAGES
%-------------------------------------------------------------------------------
\documentclass[
  a4paper, 11pt
]{MyStyle}

% improve word spacing and hyphenation
\usepackage{microtype}
\usepackage{ragged2e}
\usepackage[UTF8]{ctex}

% take care of proper font encoding
\ifxetex
	\usepackage{fontspec}
	\defaultfontfeatures{Ligatures=TeX}
% \newfontfamily\headingfont[Path = fonts/]{segoeuib.ttf} % local font
\else
	\usepackage[utf8]{inputenc}
	\usepackage[T1]{fontenc}
% \usepackage[sfdefault]{noto} % use noto google font
\fi

% enable mathematical syntax for some symbols like \varnothing
\usepackage{amssymb}

% bubble diagram configuration
\usepackage{smartdiagram}
\smartdiagramset{
  % defaut font size is \large, so adjust to harmonize with sidebar layout
  bubble center node font = \footnotesize,
  bubble node font = \footnotesize,
  % default: 4cm/2.5cm; make minimum diameter relative to sidebar size
  bubble center node size = 0.4\sidebartextwidth,
  bubble node size = 0.25\sidebartextwidth,
  distance center/other bubbles = 1.5em,
  % set center bubble color
  bubble center node color = maincolor!70,
  % define the list of colors usable in the diagram
  set color list = {maincolor!10, maincolor!40,
  maincolor!20, maincolor!60, maincolor!35},
  % sets the opacity at which the bubbles are shown
  bubble fill opacity = 0.8,
}


%-------------------------------------------------------------------------------
%                            PERSONAL INFORMATION
%-------------------------------------------------------------------------------
% profile picture
\cvprofilepic{pics/mr_hao.jpg}
% your name
\cvname{吕世豪}
% job title/career
\cvjobtitle{2021届硕士毕业生}
% date of birth
\cvbirthday{1995年3月4日}
% short address/location, use \newline if more than 1 line is required
\cvaddress{河南新乡}
% phone number
\cvphone{+86 186 2132 4369}
% email address
\cvmail{767357523@qq.com}
% pgp key
\cvlocation{上海}

% add more profile sections to sidebar on first page
\addtofrontsidebar{
	% include gosquare national flags from https://github.com/gosquared/flags;
	% naming according to ISO 3166-1 alpha-2 country codes
    \profilesection{外语能力}
	    \barskill{\faBalanceScale}{大学英语四级}{70}

    \profilesection{编程能力}
        \begin{figure}\centering
            \smartdiagram[bubble diagram]{
                \textcolor{white}{\textbf{编程}} \\ 
                \textcolor{white}{\textbf{能力}}, 
                \textcolor{black!90}{Python},
                \textcolor{black!90}{Java},
                \textcolor{black!90}{C++},
                \textcolor{black!90}{SQL},
                \textcolor{black!90}{H5}
            }
        \end{figure}
		\pointskill{\faCompress}{Django}{5}
		\pointskill{\faCompress}{SQLite}{5}
		\pointskill{\faCompress}{SQLAlchemy}{5}
		\pointskill{\faCompress}{Flask}{4}
		\pointskill{\faCompress}{Docker}{4}
		\pointskill{\faCompress}{Spark}{4}
        \pointskill{\faCompress}{VUE}{4}
}

%-------------------------------------------------------------------------------
%                         TABLE ENTRIES RIGHT COLUMN
%-------------------------------------------------------------------------------
\begin{document}

\makefrontsidebar

\cvsection{教育背景}

    \subsection{本科经历}
    \begin{cvtable}
        \cvitem{2014 -- 2018}{辽宁工程技术大学}{}{}
        \cvitem{}{}{}{}
        \cvitem{2014 -- 2016}{理科实验班}{}
            {全校前30进入实验班。主修课程Matlab、数值分析、高等代数、数学分析等,           为培养计算机编程能力打下坚实基础。}
        \cvitem{2016 -- 2018}{计算机科学与技术}{}
            {主修课程Java、数据库原理、数据结构、操作系统、计算机组成原理
            等。奠定了坚实的计算机原理知识基础。}
    \end{cvtable}

    \subsection{研究生经历}
    \begin{cvtable}
        \cvitem{2018 -- 2021}{上海海事大学}{}
            {主修课程算法与复杂性、计算机网络与数据通讯、数字图像处理、模式识别、物联网
            技术、量子智能计算等。培养了个人自主学习能力,并结合项目实践进一步综合提升了
            个人自主研发能力。}
    \end{cvtable}

\cvsection{项目经验}
    \begin{cvtable}
        \cvitem{2017 -- 2018}{人脸识别签到系统开发}{个人自主研发}
            {该项目基于大学生上课考勤状况,而针对与人脸识别模型的学习与研究,从而
            开发人脸识别系统。使用到的技术有Haar+Adaboost、CNN、CART、Django、
            SQLite、SQLAlchemy、Docker、Tensorflow等}
        \cvitem{2018 -- 2021}{特殊生物资源检测与溯源技术研究}{国家重点研发}
            {该项目为国家重点研发项目,作为主要参与人员,负责该项目的特殊生物X射线
            特征提取与识别工作。使用到的技术有Faster-RCNN、YOLO、RandomForest、
            Pytorch等}
        \cvitem{2018 -- 2019}{基于Java开发电商项目系统}{个人自主研发}
            {该项目采用分布式微服务架构,分为众多子系统。使用到的技术有SpringBoot、
            SpringCloud、Maven、Flask、Spark等}
    \end{cvtable}

\cvsection{所持证书}
    \begin{cvtable}
        \cvitem{2013年}{机动车驾驶证C1}{}{}
        \cvitem{2017年}{“蓝桥杯”省赛三等奖}{}{}
        \cvitem{2018年}{上海海事大学校级三等奖}{}{}
        \cvitem{2018年}{IEEE生物特征冬令营优秀学员}{}{}
        \cvitem{2019年}{人工智能大会小车竞赛一等奖}{}{}
    \end{cvtable}

\cvsection{个人评价}
    {具有自主学习能力,探索未知精神,热爱集体,踏实肯干。}

\end{document}